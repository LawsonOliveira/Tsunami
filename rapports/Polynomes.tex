\documentclass{article}
\usepackage[utf8]{inputenc}
\usepackage{amsmath}
\usepackage{amsfonts}

\title{Idées polynomes}
\author{antieresteban}
\date{January 2022}

\begin{document}
\section{Cas en 1D :}
Dans le cas 1D, on a eu l'idée d'utiliser les polynomes de Vandermonde, avec le problème d'ajouter un point pour ne pas avoir comme solution le polynome nul
\begin{equation*}
    \begin{pmatrix}
    1 & x_0 & \dots & x_{0}^{n} \\
    \vdots & & \vdots \\
    1 & x_{n-1} & \dots & x_{n-1}^{n} \\
    1 & x_b & \dots & x_b^{n-1} 
    \end{pmatrix}
    \begin{pmatrix}
    a_0\\
    \vdots\\
    a_{n-1}\\
    a_{n}
    \end{pmatrix}
    =
    \begin{pmatrix}
    0\\
    \vdots\\
    0\\
    1
    \end{pmatrix}
\end{equation*}
avec $x_b = \frac{1}{n}\sum_{i=0}^{n-1} x_i$\\
Cela permet d'avoir un système de Vandermonde résolvable en inversant la matrice.


\section{Cas en 2D :}
Les polynômes que nous avons trouvé à la fin de la séance sont :

\subsection{Prosition de Gabriel :}
\begin{equation*}
    F_1(x,y) = \prod_{k=0}^{n-1}((x-x_k)^2 + (y-y_k)^2)
\end{equation*}

\subsection{Proposition de Mariem :}
\begin{equation*}
    F_2(x,y) = \prod_{k=0}^{n-1}((x-x_k) + i(y-y_k))
\end{equation*}
Si l'on passe au module $P_2$, on retrouve l'expression de $P_1$, il est plus rapide de trouver les coefficients de $P_2$ mais le calcul de $P_2(x,y)$ après ça demande plus de temps (polynome dans$\mathbb{C}$).\\

\subsection{Proposition matricielle :}
\subsubsection*{Début d'idée}
\begin{equation*}
    F_3(x,y) = \sum_{i,j = 0}^{n} \alpha_{ij}x^iy^j
\end{equation*}
\begin{equation*}
    \begin{pmatrix}
    1 & x_0 & y_0 & x_0y_0 & \dots & x_{0}^{n}y_{0}^{n} \\
    \vdots & & & & \vdots \\
    1 & x_{n-1} & y_{n-1} & x_{n-1}y_{n-1} & \dots & x_{n-1}^{n}y_{n-1}^{n} \\
    1 & x_{b} & y_{b} & x_{b}y_{b} & \dots & x_{b}^{n}y_{b}^{n} \\
    \end{pmatrix}
    \begin{pmatrix}
    a_0\\
    \vdots\\
    a_{n-1}\\
    a_{n}
    \end{pmatrix}
    =
    \begin{pmatrix}
    0\\
    \vdots\\
    0\\
    1
    \end{pmatrix}
\end{equation*}
Où $x_b$ et $y_b$ sont définis de manière similaire à $x_b$ au début du document. Dans cette méthode il nous faut inverser la matrice, contrairement au cas en 1D avec de vraies matrices de Vandermonde cela n'est pas théoriquement assuré.

\subsubsection*{L'implémentation}
%Soit $(p_i)_{i \in \llbracket chips \rrbracket}$
La fonction principale est set polynome xpy numpy matrix.
Elle prend en entrée coords, un array de coordonnées où le polynôme à construire est nul.\\

Etape 1 : Construire matrice avec u
\begin{equation}
    \begin{pmatrix}
    A_{n} = 
\end{pmatrix}
\end{equation}
\end{document}
