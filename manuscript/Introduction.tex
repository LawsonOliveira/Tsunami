\chapter{Introduction }\label{ch:Introduction}
\section{Context and objectifs}\label{sec:Contexte}
Tsunamis do not occur very often but are devastating phenomena. Research
and studies about tsunamis try to predict the impact of such events to protect
people. Due to the sheer scale of tsunami, laboratory and analytical models are
not relevant. The former because we can not obtain an equivalent scale model
in a lab, the latter because there is no analytical solution to the propagation
equations. Although a numerical resolution can approximate the solution, it
requires a huge computing power and time without taking into account the accuracy errors due to turbulence phenomena. 

\section{State of the art}\label{sec:Contexte}



\section{Contribution}\label{sec:Contribution}


\section{Report Structure}\label{sec:Contribution}
First, it is presented how important and effective this approach is, then
the second chapter shows basic concepts about a neural network and its behavior, as well as a brief introduction to differential equations.
Then in the third paragraph we explain our approche and solve some partial differential equations using neural networks.
After that, in the fifth paragraph, the describes how obtain the domain to do the Tsunami simulation, that is, the mesh to make the simulation.